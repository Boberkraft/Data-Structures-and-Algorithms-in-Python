\documentclass{article}
\usepackage{graphicx}
\graphicspath{{/}}
\begin{document}

In Section 4.2 we prove by induction that the number of lines printed by a call to $interval(c)$ is $2^{c} - 1$. Another interesting question is how many dashes are printed during that process. Prove by induction that the number of dashes printed by $interval(c)$ is $2^{c+1} -c-2$.\\\\

$interval(n) = 2^{c+1}-c-2 $\\
Podstawa:\\
$n > 0$\\
$interval(1) = 2^{2}-1-2 = 1$\\
$interval(2) = 2^{3}-2-2 = 4$\\\\

Hipoteza:
 Zauwazylem, ze kazdy nowy przedzial skladal sie z: \\
$\bullet $ jednego mniejszego przedzialu $n-1$, \\
$\bullet $ duzej kreski o dlugosci $n$, \\
$\bullet $ i znow mniejszego przedzialu $n-1$\\\\
$interval(n+1) = 2\cdot interval(n)+n+1$\\\\
$= 2\cdot(2^{n+1}-1) + n + 1$\\
$= 2\cdot 2^{n+1}- 2 +n+1$\\
$= 2^{(n+1)+1}-(n+1)-2 $\\\\


\begin{center}
	\includegraphics[scale=0.5]{birb}
\end{center}
\end{document}
\documentclass{article}

\usepackage{amsmath}
\usepackage{amssymb}

\begin{document}
Consider the following “justification” that the Fibonacci function, $F(n)$ (see Proposition 3.20) is $O(n)$: \\
Base case ($n \geq 2$):\\
$F(1) = 1$ and $F(2) = 2$.\\
Induction step $(n > 2)$:\\
Assume claim true for $n' < n$. Consider $n$. \\$ F(n) = F(n-2) +F(n-1)$. \\By induction, $F(n-2)$ is $O(n-2)$ and $F(n-1)$ is $O(n-1)$.\\ Then, $F(n)$ is $O((n-2)+(n-1))$, by the identity presented in Exercise R-3.11. Therefore, $F(n)$ is $O(n)$.\\
What is wrong with this “justification”?\\\\\\
 
I will show by example that Fibonacci function, $F(n)$ is not $O(n)$.\\
For now lets try to proof that $F(n)$ is $O(n)$ and that means:
\begin{center}
	 $F(n) \geq c \cdot n$, for $c = \{c \in \mathbb{R} \mid c > 0 \}$, $ n \geq 1 $.\\
\end{center} 
Firstly we need to find a constant that matches to our function.\\
$F(1) = 1$\\
$F(2) = 1$\\
$F(3) = 2$\\
$F(4) = 3$\\
Let $c$ be 120.\\
Base cases:
\begin{center}
	$
	\begin{array}{cccc}
	n	& F(n) & & 120n \\ 
	\hline
	1& 1 &<& 120 \\ 
	2& 1 &<& 230 \\ 
	3& 2 &<& 360 \\ 
	\end{array} 
	$
\end{center}
Inductive hipotessis:\\
$F(k) \geq ck$ for some $k \geq n_0$, \\and we want to show $F(k) \geq ck$ for every k\\

\begin{align*}
		F(k) &= F(k-1) + F(k-2)\\
			&\leq c(k-1) + c(k-2), \text{ by inductive hipotessis}\\
			&=c(k+k-1 - 2) \\
			&=c(2k - 3) \\
			&=2ck - 3c \\
			&< 2ck \text{, since }0 > -3c \\
			&\nleq ck
\end{align*}
This shows that $F(n)$ can not be $O(n)$ because i have proved \\ that by example? $ \blacksquare $
\end{document}
\usepackage{polski}
\usepackage[utf8]{inputenc}
\usepackage{graphicx}
\usepackage{hyperref}
\graphicspath{{/}}
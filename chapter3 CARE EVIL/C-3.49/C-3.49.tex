\documentclass{article}
\usepackage{graphicx}
\graphicspath{ {/} }
\usepackage{setspace}
\doublespacing
\begin{document}

Consider the Fibonacci function, F(n) (see Proposition 3.20). Show by
induction that $F(n)$ is $\Omega((3/2)^n)$.\\\\

Lets find $c$ and $n_{0}$ such that $F(n) \geq c(\frac{3}{2})^{n}$ for constant c and n $\geq n_0$\\
for $c = \frac{4}{9}$ and $n_{0} = 1$ everything works great\\
\begin{center}
	$
	\begin{array}{ccc}
	n	& F(n) & \frac{4}{9} (\frac{3}{2})^n \\ 
	1	& 1 & \frac{2}{3} \\ 
	2	& 1 & 1 \\ 
	3	& 2 & \frac{3}{2} \\
	4	& 3 & 2\frac{1}{4}
	\end{array} 
	$
\end{center}
Inductive Hippotesis:\\
$F(k) \geq c(\frac{3}{2})^{k}$, for some $k \geq n_0$\\
Lets schow that $F(k+1) \geq c(\frac{3}{2})^{k+1}$\\

$F(k+1) = F(k) + F(k-1)$\\
Lets apply out Hippotesis\\
\begin{math}
F(k+1) \geq c(\frac{3}{2})^{k} + c(\frac{3}{2})^{k-1}\\
= c(\frac{3}{2})^{k-1}\cdot\frac{3}{2} + c(\frac{3}{2})^{k-1}\\
= c(\frac{3}{2})^{k-1}(\frac{3}{2}+1)\\
> c(\frac{3}{2})^{k-1}(\frac{3}{2})^{2}, \textnormal{ since } \frac{5}{2} > \frac{9}{4} \\
= c(\frac{3}{2})^{k+1}
\end{math}

\begin{center}
	\includegraphics[scale=0.3]{birb}
	http://www.simonfoucher.com/McGill/COMP250/Lectures/Math/lecture13.pdf
\end{center}

	
\end{document}


= F(k+1) + F(k-1)\\
\geq c(\frac{3}{2})^{k} + c(\frac{3}{2})^{k-1}\\
= c(\frac{3}{2})^{k-1}\cdot\frac{3}{2} + c(\frac{3}{2})^{k-1}\\
= c(\frac{3}{2})^{k-1}(\frac{3}{2}+1)\\
> c(\frac{3}{2})^{k-1}(1+1)\\
= c((\frac{3}{2})^{k-1}+ (\frac{3}{2})^{k-1})\\
= c((2\cdot (\frac{3}{2})^{k-1})\\
> c((\frac{3}{2}\cdot (\frac{3}{2})^{k-1})\\
= c(\frac{3}{2})^{k}\\

oohh god XD